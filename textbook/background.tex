\subsection{Bits, Bytes, and their Representations}
\subsubsection{Numbers in different bases}
Bits are the fundamental unit of data on a computer. A bit can only be either on
or off, 0 or 1. It's awkward to represent data in terms of bits, so they are
usually referred to in groups. A string of eight consecutive bits is called a
byte, and a pair of two bytes, or 16 bits, is called a word. Bytes are often
further grouped into pairs, called double words, or groups of four, called quad
words.

Since a bit can only take on one of two values, computers store numbers in base
two, or binary. Just as the digits of a number in base 10 are each scaled by a
power of 10, each bit in a binary number is scaled by a power of 2. The
rightmost bit has a value of either 0 or 1 (scaled by $2^0$, or 1), and every
other bit is scaled by twice as much as the bit to its right. Therefore, if we
zero-index the bits starting from the right, the $i$th bit is scaled by $2^i$.

\begin{exmp}[Numbers in base 2]
\begin{align*}
11010110_2 & = 1 \times 2^7 + 1 \times 2^6 + 0 \times 2^5 + 1 \times 2^4 + 0 \times 2^3 + 1 \times 2^2 + 1 \times 2^1 + 0 \times 2^0 \\
           & = 128 + 64 + 0 + 16 + 0 + 4 + 2 + 0 \\
           & = 214
\end{align*}
Note that the 2 subscript denotes a number written in base 2.
\end{exmp}

Since it's tedious to write bytes as strings of bits, they are often represented
in base 16, or hexadecimal. This representation is convenient since 4 bits can
be represented with a single hexadecimal digit. Since there are more hexadecimal
digits than decimal ones, we use a-f as digits with values 10-15.

\begin{exmp}[Numbers in base 16]
\begin{align*}
11010110_2 &= 1101_2 \times 16^1 + 0110_2 \times 16^0 \\
           &= 13 \times 16^1 + 6 \times 16^0 \\
           &= \emph{0xd6}
\end{align*}
Note that the 0x prefix denotes a number written in base 16.
\end{exmp}

\subsubsection{2s complement}
Since computers can only store data as bits, there is no inherent way to
represent negative numbers. To address this problem, the highest-order bit is
given a negative value when negative numbers are needed. This representation for
negative numbers is called 2's complement. Whereas a string of $n$ bits normally
takes values from $0$ to $2^n-1$, the same $n$ bits in 2's complement can take
any values from $-2^{n-1}$ to $2^{n-1}-1$.

\subsubsection{Machine Words}
The number of bits that a computer can read, write, and manipulate at a time is
called a machine word, not to be confused by the 16-bit words from above. A
computer that operates on 16 bits at a time is said to run on a 16-bit
architecture. The size of a machine word varies between computers. At the time
of writing, most modern computers have 64-bit machine words, and thus run on
64-bit architectures. The size of machine words generally gets smaller as the
computer gets older. The original PlayStation and the GameCube ran on 32-bit
architectures, and the original GameBoy had an 8-bit architecture. x86 is the
most common 32-bit architecture, and it's successor x86-64, is the most common
64-bit architecture.

\subsubsection{Endianness}
Not all computers store multiple bytes of data in the same order. Some store the
most significant byte first, which results in a number like
$\texttt{0x080485a2}$ being stored as $\texttt{0x08}$ $\texttt{0x04}$
$\texttt{0x85}$ $\texttt{0xa2}$. This is called big-endian byte order, and it is
surprisingly rare. Most computers store data in little-endian byte order, which
lists the least-significant byte first. The same number $\texttt{0x080485a2}$
stored in little-endian byte order would be stored as $\texttt{0xa2}$
$\texttt{0x85}$ $\texttt{0x04}$ $\texttt{0x08}$.

% cpu, registers, and memory
\subsection{Computer Model}
Although we tend to think of a ``computer'' as consisting of many parts such as a
monitor, hard disk, mouse, CD drive, etc., there are only three components we
need to know about in order to exploit software.

\subsubsection{The CPU}
The Central Processing Unit, or CPU, is responsible for executing the
instructions contained in a program. This typically includes performing
arithmetic, reading and writing to memory, and making requests to the kernel via
syscalls. Different CPUs understand different variants of machine code, and a
CPU can only run an executable if it is written in the variant that the CPU
understands.

\subsubsection{Memory}
Memory acts as both a scratchpad for the CPU to use while executing a program,
and the place where the CPU reads program instructions. It is also used to keep
track of function calls and handle recursion. Memory is the \emph{only} place
where the CPU can read and write data.

\subsubsection{Registers}
Registers are very fast memory located on the CPU. Although they are fast, each
register can typically only store a single machine word, which means the vast
majority of data must reside in main memory. 

\subsubsection{Compilers}
A \emph{compiler} translates source code into machine code, producing an object
file. An object file cannot be run until it is linked, a task which is left
to the linker. 

\subsubsection{Linkers}
Linkers combine object files in a process called linking. This produces an
executable, a binary which can be executed. This may sound confusing since we
typically say that we run binaries after compiling them, but what programmers
colloquially refer to as ``compiling'' is actually compiling \textit{and}
linking.
