By now you already know enough to start learning about and exploiting bugs, but
you'll quickly find that a lot of the difficulty comes from figuring out how to
do the things you want to. Interacting with a program, manipulating data, and
finding the information you need used to involve mastering several different
ad-hoc utilities. Now we have access to tools that were developed with
exploitation in mind to make these small tasks as easy as possible so you can
focus only on finding your exploit.

\subsection{\texttt{pwntools}}
\texttt{pwntools} is an exploit development framework written for python. It
consists of several modules, each of which is designed to make a specific task
much easier.  pwntools has over 40 differet modules at the time of writing, but
we'll only go over a few of the most useful ones here.

\begin{itemize}
    \item \texttt{pwnlib.tubes} - sending and receiving data
    \item \texttt{pwnlib.util} - manipulating data
\end{itemize}

\subsubsection{\texttt{pwnlib.tubes}}
The \texttt{tubes} module is used for communication between your program and
just about everything else. It has a number of submodules each of which are used
to communicate over different interfaces. For example,
\texttt{pwnlib.tubes.process} can communicate with any program you can run
locally. Here is an example using some common shell utilities.

\begin{lstlisting}
>>> p = process("cat")
>>> p.send("hello")
>>> p.read()
'hello'
>>> p.sendline("goodbye")
>>> p.readuntil("by")
'goodby'
>>> p.readline()
'e\n'
>>> p.shutdown()
>>> p = process(["echo", "test", "string"]) 
>>> p.readregex(".*str")
'test str'
>>> p.readline()
'ing\n'
\end{lstlisting}

\texttt{pwnlib.tubes.remote} is used to make a TCP or UDP connection to a remote
host. It's counterpart, \texttt{pwnlib.tubes.listen} is used to receive those
connections. An example is given below.

\begin{lstlisting}
>>> l = listen(8888)
>>> r = remote("localhost", 8888)
>>> r.sendline("hello from remote")
>>> l.readline()
'hello from remote\n'
>>> l.send("hello from listen")
>>> r.readuntil("from")
'hello from'
>>> r.read()
' listen'
>>> r.send("{\"length\":\"7\", \"type\":\"ascii\", \"data\":\"network\"}")
>>> l.readregex("\"type\":\".*\"")
'{"length":"7", "type":"ascii"'
\end{lstlisting}

You'll notice that the \texttt{remote} and \texttt{listen} classes have some of
the same methods as the \texttt{process} class. This is the beauty of the
\texttt{tubes} module; it provides a single interface which can be used to
communicate with multiple external processes. There are also submodules for
serial ports and ssh communication, both of which use the same methods in the
examples above. These methods are described in more detail below.

\begin{itemize}
    \item \texttt{send} - sends data
    \item \texttt{sendline} - like send, but appends a newline before sending
    \item \texttt{read} / \texttt{recv} - receive data
    \item \texttt{readline} / \texttt{recvline} - receive data until a newline is found and return
        it
    \item \texttt{readuntil} / \texttt{recvuntil} receive data until the specified string is found,
        and return it
    \item \texttt{readregex} / \texttt{recvregex} - receive data until the specified regex is
        matched, then return it
\end{itemize}

\subsubsection{\texttt{pwnlib.util}}
The util module provides functions for several of the tasks that
frequently come up when developing exploits. Submodules include the
\texttt{fiddling} module for bit operations and converting between different
formats, the \texttt{lists} module for manipulating lists, and the
\texttt{packing} module for converting between numbers and their equivalent
data strings.

\begin{lstlisting}
>>> s = "pwntools is awesome"
>>> s_hex = enhex(s)
>>> s_hex
'70776e746f6f6c7320697320617765736f6d65'
>>> print hexdump(s)
00000000  70 77 6e 74  6f 6f 6c 73  20 69 73 20  61 77 65 73  |pwnt│ools│ is │awes|
00000010  6f 6d 65                                            |ome|
00000013
>>> unhex(s_hex) == s
True
>>>
>>> s_ord = ordlist(s)
>>> s_ord
[112, 119, 110, 116, 111, 111, 108, 115, 32, 105, 115, 32, 97, 119, 101, 115, 111, 109, 101]
>>> s_ord[0] += 1
>>> s = unordlist(s_ord)
>>> s
'qwntools is awesome'
>>>
>>> val = 0xffffdeadbeefffff
>>> data = p64(val)
>>> data
'\xff\xff\xef\xbe\xad\xde\xff\xff'
>>> val == u64(data)
True
\end{lstlisting}

As before, here is a description of each of the functions and their uses.
\begin{itemize}
    \item \texttt{enhex} - convert an ascii string to hexadecimal
    \item \texttt{unhex} - convert a hexadecimal string to ascii
    \item \texttt{hexdump} - return a hex dump of a hexadecimal string
    \item \texttt{ordlist} - convert a string to a list of ascii values
    \item \texttt{unordlist} - convert a list of ascii values to a string
    \item \texttt{p64} - pack an integer into a 64-bit machine word
    \item \texttt{u64} - unpack a 64-bit machine word into an integer
\end{itemize}

\subsection{\texttt{pwndbg}}
Although \texttt{gdb} is nice, it makes it unnecessarily difficult to find some
of the information we need. \texttt{pwndbg} is an extension to \texttt{gdb}
written with exploit development in mind. In addition to adding several useful
commands, it also automatically displays program information at every
breakpoint.

You can activate \texttt{pwntools} by either running \texttt{source
<path\_to\_pwntools>} in \texttt{gdb} or by putting the line in
\texttt{~./gdbinit}. Below is the output after opening \texttt{elf\_example.c} in
\texttt{pwndbg}.

\begin{lstlisting}
> gdb elf
GNU gdb (Ubuntu 7.11.1-0ubuntu1~16.5) 7.11.1
Copyright (C) 2016 Free Software Foundation, Inc.
License GPLv3+: GNU GPL version 3 or later <http://gnu.org/licenses/gpl.html>
This is free software: you are free to change and redistribute it.
There is NO WARRANTY, to the extent permitted by law.  Type "show copying"
and "show warranty" for details.
This GDB was configured as "x86_64-linux-gnu".
Type "show configuration" for configuration details.
For bug reporting instructions, please see:
<http://www.gnu.org/software/gdb/bugs/>.
Find the GDB manual and other documentation resources online at:
<http://www.gnu.org/software/gdb/documentation/>.
For help, type "help".
Type "apropos word" to search for commands related to "word"...
Loaded 112 commands. Type pwndbg [filter] for a list.
Reading symbols from elf...(no debugging symbols found)...done.
pwndbg> 
\end{lstlisting}

As explained in the help message, you can run \texttt{pwndbg} to display all of
the commands added by \texttt{pwndbg}, and optionally add a filter to display
only the commands that match the filter. All of the commands from vanilla
\texttt{gdb} are still available in \texttt{pwndbg}. This means you can still
use \texttt{ni}, \texttt{si}, \texttt{x}, etc., as usual.


The best way to see what \texttt{pwndbg} has to offer is to just start using it.
Set a breakpoint at \texttt{main()} and stop at it.

\begin{lstlisting}
pwndbg> b main
Breakpoint 1 at 0x4005fa
pwndbg> r
Starting program: /home/devneal/Security/REFE/textbook/example_code/elf 

Breakpoint 1, 0x00000000004005fa in main ()
LEGEND: STACK | HEAP | CODE | DATA | RWX | RODATA
[───────────REGISTERS───────────]
*RAX  0x4005f6 (main) ◂— push   rbp
 RBX  0x0
 RCX  0x0
*RDX  0x7fffffffdfb8 —▸ 0x7fffffffe2ee ◂— 'XDG_VTNR=7'
*RDI  0x1
*RSI  0x7fffffffdfa8 —▸ 0x7fffffffe2b8 ◂— 0x65642f656d6f682f ('/home/de')
*R8   0x4006c0 (__libc_csu_fini) ◂— ret    
*R9   0x7ffff7de7ab0 (_dl_fini) ◂— push   rbp
*R10  0x846
*R11  0x7ffff7a2d740 (__libc_start_main) ◂— push   r14
*R12  0x400500 (_start) ◂— xor    ebp, ebp
*R13  0x7fffffffdfa0 ◂— 0x1
 R14  0x0
 R15  0x0
*RBP  0x7fffffffdec0 —▸ 0x400650 (__libc_csu_init) ◂— push   r15
*RSP  0x7fffffffdec0 —▸ 0x400650 (__libc_csu_init) ◂— push   r15
*RIP  0x4005fa (main+4) ◂— sub    rsp, 0x10
[───────────DISASM───────────]
 ► 0x4005fa <main+4>     sub    rsp, 0x10
   0x4005fe <main+8>     mov    rax, qword ptr fs:[0x28]
   0x400607 <main+17>    mov    qword ptr [rbp - 8], rax
   0x40060b <main+21>    xor    eax, eax
   0x40060d <main+23>    mov    edi, 0x4006d4
   0x400612 <main+28>    call   puts@plt                      <0x4004b0>
 
   0x400617 <main+33>    lea    rax, [rbp - 0xc]
   0x40061b <main+37>    mov    rsi, rax
   0x40061e <main+40>    mov    edi, 0x4006e0
   0x400623 <main+45>    mov    eax, 0
   0x400628 <main+50>    call   __isoc99_scanf@plt            <0x4004e0>
[────────────STACK────────────]
00:0000│ rbp rsp  0x7fffffffdec0 —▸ 0x400650 (__libc_csu_init) ◂— push   r15
01:0008│          0x7fffffffdec8 —▸ 0x7ffff7a2d830 (__libc_start_main+240) ◂— mov    edi, eax
02:0010│          0x7fffffffded0 ◂— 0x0
03:0018│          0x7fffffffded8 —▸ 0x7fffffffdfa8 —▸ 0x7fffffffe2b8 ◂— 0x65642f656d6f682f ('/home/de')
04:0020│          0x7fffffffdee0 ◂— 0x1f7ffcca0
05:0028│          0x7fffffffdee8 —▸ 0x4005f6 (main) ◂— push   rbp
06:0030│          0x7fffffffdef0 ◂— 0x0
07:0038│          0x7fffffffdef8 ◂— 0xbfb8facd527a9cd
[───────────BACKTRACE───────────]
 ► f 0           4005fa main+4
   f 1     7ffff7a2d830 __libc_start_main+240
Breakpoint main
pwndbg> 
\end{lstlisting}

\texttt{pwndbg} will automatically display the contents of the registers, the
disassembled instructions around \texttt{rip}, the top of the stack, and a
backtrace giving the current chain of function calls. You can display this
information at any time with the \texttt{context} command. If you only want to
see one of the sections, run \texttt{context} followed by the name of the
section.

\texttt{pwndbg} provides the \texttt{nearpc} and \texttt{emulate} commands for
displaying commands near \texttt{\$rip} (also called the program counter).
Note that these commands will only display commands that will be executed based
on the current state of the program. This makes a significant difference when
analyzing code with lots of jump instructions. Pause execution at the call to
\texttt{puts()}.

\begin{lstlisting}
pwndbg> nearpc
 ► 0x4005fa <main+4>     sub    rsp, 0x10
   0x4005fe <main+8>     mov    rax, qword ptr fs:[0x28]
   0x400607 <main+17>    mov    qword ptr [rbp - 8], rax
   0x40060b <main+21>    xor    eax, eax
   0x40060d <main+23>    mov    edi, 0x4006d4
   0x400612 <main+28>    call   puts@plt                      <0x4004b0>
 
   0x400617 <main+33>    lea    rax, [rbp - 0xc]
   0x40061b <main+37>    mov    rsi, rax
   0x40061e <main+40>    mov    edi, 0x4006e0
   0x400623 <main+45>    mov    eax, 0
   0x400628 <main+50>    call   __isoc99_scanf@plt            <0x4004e0>
pwndbg> break *main+28
Breakpoint 2 at 0x400612
pwnbdg> c
Continuing.

0x0000000000400612 in main ()
LEGEND: STACK | HEAP | CODE | DATA | RWX | RODATA
[───────────REGISTERS───────────]
*RAX  0x0
 RBX  0x0
 RCX  0x0
 RDX  0x7fffffffdfb8 —▸ 0x7fffffffe2ee ◂— 'XDG_VTNR=7'
*RDI  0x4006d4 ◂— and    byte ptr [rbp + 0x78], r12b /* 'ELF example' */
 RSI  0x7fffffffdfa8 —▸ 0x7fffffffe2b8 ◂— 0x65642f656d6f682f ('/home/de')
 R8   0x4006c0 (__libc_csu_fini) ◂— ret    
 R9   0x7ffff7de7ab0 (_dl_fini) ◂— push   rbp
 R10  0x846
 R11  0x7ffff7a2d740 (__libc_start_main) ◂— push   r14
 R12  0x400500 (_start) ◂— xor    ebp, ebp
 R13  0x7fffffffdfa0 ◂— 0x1
 R14  0x0
 R15  0x0
 RBP  0x7fffffffdec0 —▸ 0x400650 (__libc_csu_init) ◂— push   r15
*RSP  0x7fffffffdeb0 —▸ 0x7fffffffdfa0 ◂— 0x1
*RIP  0x400612 (main+28) ◂— call   0x4004b0
[───────────DISASM───────────]
   0x4005fa <main+4>     sub    rsp, 0x10
   0x4005fe <main+8>     mov    rax, qword ptr fs:[0x28]
   0x400607 <main+17>    mov    qword ptr [rbp - 8], rax
   0x40060b <main+21>    xor    eax, eax
   0x40060d <main+23>    mov    edi, 0x4006d4
 ► 0x400612 <main+28>    call   puts@plt                      <0x4004b0>
        s: 0x4006d4 ◂— 'ELF example'
 
   0x400617 <main+33>    lea    rax, [rbp - 0xc]
   0x40061b <main+37>    mov    rsi, rax
   0x40061e <main+40>    mov    edi, 0x4006e0
   0x400623 <main+45>    mov    eax, 0
   0x400628 <main+50>    call   __isoc99_scanf@plt            <0x4004e0>
[────────────STACK────────────]
00:0000│ rsp  0x7fffffffdeb0 —▸ 0x7fffffffdfa0 ◂— 0x1
01:0008│      0x7fffffffdeb8 ◂— 0x4e5d491772575800
02:0010│ rbp  0x7fffffffdec0 —▸ 0x400650 (__libc_csu_init) ◂— push   r15
03:0018│      0x7fffffffdec8 —▸ 0x7ffff7a2d830 (__libc_start_main+240) ◂— mov    edi, eax
04:0020│      0x7fffffffded0 ◂— 0x0
05:0028│      0x7fffffffded8 —▸ 0x7fffffffdfa8 —▸ 0x7fffffffe2b8 ◂— 0x65642f656d6f682f ('/home/de')
06:0030│      0x7fffffffdee0 ◂— 0x1f7ffcca0
07:0038│      0x7fffffffdee8 —▸ 0x4005f6 (main) ◂— push   rbp
[───────────BACKTRACE───────────]
 ► f 0           400612 main+28
   f 1     7ffff7a2d830 __libc_start_main+240
pwndbg> 
\end{lstlisting}

Notice how \texttt{pwndbg} will automatically display the arguments to the call
to \texttt{puts()}, and print it as a string. \texttt{pwndbg} offers several
ways to display this information. \texttt{db} will display data as bytes,
\texttt{dw} will display it as words, \texttt{dd} will display it as double
words, \texttt{dq} will display it as quad words, and \texttt{ds} will display
it as a string.

\begin{lstlisting}
pwndbg> db 0x4006d4
00000000004006d4     45 4c 46 20 65 78 61 6d 70 6c 65 00 25 64 00 00
00000000004006e4     01 1b 03 3b 30 00 00 00 05 00 00 00 bc fd ff ff
00000000004006f4     7c 00 00 00 1c fe ff ff 4c 00 00 00 12 ff ff ff
0000000000400704     a4 00 00 00 6c ff ff ff c4 00 00 00 dc ff ff ff
pwndbg> dw 0x4006d4
00000000004006d4     4c45 2046 7865 6d61 6c70 0065 6425 0000
00000000004006e4     1b01 3b03 0030 0000 0005 0000 fdbc ffff
00000000004006f4     007c 0000 fe1c ffff 004c 0000 ff12 ffff
0000000000400704     00a4 0000 ff6c ffff 00c4 0000 ffdc ffff
pwndbg> dd 0x4006d4
00000000004006d4     20464c45 6d617865 00656c70 00006425
00000000004006e4     3b031b01 00000030 00000005 fffffdbc
00000000004006f4     0000007c fffffe1c 0000004c ffffff12
0000000000400704     000000a4 ffffff6c 000000c4 ffffffdc
pwndbg> dq 0x4006d4
00000000004006d4     6d61786520464c45 0000642500656c70
00000000004006e4     000000303b031b01 fffffdbc00000005
00000000004006f4     fffffe1c0000007c ffffff120000004c
0000000000400704     ffffff6c000000a4 ffffffdc000000c4
pwndbg> ds 0x4006d4
4006d4 'ELF example'
\end{lstlisting}

If you have a list of pointers in memory, you can resolve several of them at
once with \texttt{dps} or \texttt{telescope}. This tends to be most useful for
the stack, so if you run the commands with no arguments it will default to the
stack.

\begin{lstlisting}
pwndbg> dps
00:0000│ rsp  0x7fffffffdeb0 —▸ 0x7fffffffdfa0 ◂— 0x1
01:0008│      0x7fffffffdeb8 ◂— 0x562ac01897d4100
02:0010│ rbp  0x7fffffffdec0 —▸ 0x400650 (__libc_csu_init) ◂— push   r15
03:0018│      0x7fffffffdec8 —▸ 0x7ffff7a2d830 (__libc_start_main+240) ◂— mov    edi, eax
04:0020│      0x7fffffffded0 ◂— 0x0
05:0028│      0x7fffffffded8 —▸ 0x7fffffffdfa8 —▸ 0x7fffffffe2bb ◂— 0x65642f656d6f682f ('/home/de')
06:0030│      0x7fffffffdee0 ◂— 0x1f7ffcca0
07:0038│      0x7fffffffdee8 —▸ 0x4005f6 (main) ◂— push   rbp
pwndbg> 
\end{lstlisting}

Set a breakpoint at the call to \texttt{scanf()} and pause execution there.
You'll notice that the arguments are a ``\%d'' format string and a memory
address, which is currently uninitialized. At this point you have 3 breakpoints:
one at the start of \texttt{main()}, one at the call to \texttt{puts()} and one
at the call to \texttt{scanf()}. Vanilla \texttt{gdb} provides \texttt{info
breakpoints}, but \texttt{pwndbg} has a family of commands for breakpoint
control. \texttt{bl} lists breakpoints, \texttt{bc} clears them,
\texttt{be} and \texttt{bd} enables and disables them respectively, and
\texttt{bp} sets them.

\begin{lstlisting}
pwndbg> bp *main+50
Breakpoint 3 at 0x400628
pwndbg> c
Continuing.
ELF example

Breakpoint 3, 0x0000000000400628 in main ()
LEGEND: STACK | HEAP | CODE | DATA | RWX | RODATA
[───────────REGISTERS───────────]
 RAX  0x0
 RBX  0x0
*RCX  0x7ffff7b04290 (__write_nocancel+7) ◂— cmp    rax, -0xfff
*RDX  0x7ffff7dd3780 (_IO_stdfile_1_lock) ◂— 0x0
*RDI  0x4006e0 ◂— and    eax, 0x1000064 /* '%d' */
*RSI  0x7fffffffdeb4 ◂— 0x363f190000007fff
*R8   0x602000 ◂— 0x0
*R9   0xd
*R10  0x7ffff7dd1b78 (main_arena+88) —▸ 0x602410 ◂— 0x0
*R11  0x246
 R12  0x400500 (_start) ◂— xor    ebp, ebp
 R13  0x7fffffffdfa0 ◂— 0x1
 R14  0x0
 R15  0x0
 RBP  0x7fffffffdec0 —▸ 0x400650 (__libc_csu_init) ◂— push   r15
 RSP  0x7fffffffdeb0 —▸ 0x7fffffffdfa0 ◂— 0x1
*RIP  0x400628 (main+50) ◂— call   0x4004e0
[───────────DISASM───────────]
   0x400612 <main+28>    call   puts@plt                      <0x4004b0>
 
   0x400617 <main+33>    lea    rax, [rbp - 0xc]
   0x40061b <main+37>    mov    rsi, rax
   0x40061e <main+40>    mov    edi, 0x4006e0
   0x400623 <main+45>    mov    eax, 0
 ► 0x400628 <main+50>    call   __isoc99_scanf@plt            <0x4004e0>
        format: 0x4006e0 ◂— 0x3b031b0100006425 /* '%d' */
        vararg: 0x7fffffffdeb4 ◂— 0x363f190000007fff
 
   0x40062d <main+55>    mov    eax, 0
   0x400632 <main+60>    mov    rdx, qword ptr [rbp - 8]
   0x400636 <main+64>    xor    rdx, qword ptr fs:[0x28]
   0x40063f <main+73>    je     main+80                       <0x400646>
 
   0x400641 <main+75>    call   __stack_chk_fail@plt          <0x4004c0>
[────────────STACK────────────]
00:0000│ rsp rsi-4  0x7fffffffdeb0 —▸ 0x7fffffffdfa0 ◂— 0x1
01:0008│            0x7fffffffdeb8 ◂— 0xeec7bf58363f1900
02:0010│ rbp        0x7fffffffdec0 —▸ 0x400650 (__libc_csu_init) ◂— push   r15
03:0018│            0x7fffffffdec8 —▸ 0x7ffff7a2d830 (__libc_start_main+240) ◂— mov    edi, eax
04:0020│            0x7fffffffded0 ◂— 0x0
05:0028│            0x7fffffffded8 —▸ 0x7fffffffdfa8 —▸ 0x7fffffffe2bb ◂— 0x65642f656d6f682f ('/home/de')
06:0030│            0x7fffffffdee0 ◂— 0x1f7ffcca0
07:0038│            0x7fffffffdee8 —▸ 0x4005f6 (main) ◂— push   rbp
[───────────BACKTRACE───────────]
 ► f 0           400628 main+50
   f 1     7ffff7a2d830 __libc_start_main+240
Breakpoint *main+50
pwndbg> bl
Num     Type           Disp Enb Address            What
1       breakpoint     keep y   0x00000000004005f6 <main>
	breakpoint already hit 1 time
2       breakpoint     keep y   0x0000000000400612 <main+28>
	breakpoint already hit 1 time
3       breakpoint     keep y   0x0000000000400628 <main+50>
	breakpoint already hit 1 time
pwndbg> bd 1
pwndbg> r
Starting program: /home/devneal/Security/REFE/textbook/example_code/elf 

Breakpoint 2, 0x0000000000400612 in main ()
LEGEND: STACK | HEAP | CODE | DATA | RWX | RODATA
[───────────REGISTERS───────────]
 RAX  0x0
 RBX  0x0
*RCX  0x0
*RDX  0x7fffffffdfb8 —▸ 0x7fffffffe2f1 ◂— 'XDG_VTNR=7'
*RDI  0x4006d4 ◂— and    byte ptr [rbp + 0x78], r12b /* 'ELF example' */
*RSI  0x7fffffffdfa8 —▸ 0x7fffffffe2bb ◂— 0x65642f656d6f682f ('/home/de')
*R8   0x4006c0 (__libc_csu_fini) ◂— ret    
*R9   0x7ffff7de7ab0 (_dl_fini) ◂— push   rbp
*R10  0x846
*R11  0x7ffff7a2d740 (__libc_start_main) ◂— push   r14
 R12  0x400500 (_start) ◂— xor    ebp, ebp
 R13  0x7fffffffdfa0 ◂— 0x1
 R14  0x0
 R15  0x0
 RBP  0x7fffffffdec0 —▸ 0x400650 (__libc_csu_init) ◂— push   r15
 RSP  0x7fffffffdeb0 —▸ 0x7fffffffdfa0 ◂— 0x1
*RIP  0x400612 (main+28) ◂— call   0x4004b0
[───────────DISASM───────────]
   0x4005fa <main+4>     sub    rsp, 0x10
   0x4005fe <main+8>     mov    rax, qword ptr fs:[0x28]
   0x400607 <main+17>    mov    qword ptr [rbp - 8], rax
   0x40060b <main+21>    xor    eax, eax
   0x40060d <main+23>    mov    edi, 0x4006d4
 ► 0x400612 <main+28>    call   puts@plt                      <0x4004b0>
        s: 0x4006d4 ◂— 'ELF example'
 
   0x400617 <main+33>    lea    rax, [rbp - 0xc]
   0x40061b <main+37>    mov    rsi, rax
   0x40061e <main+40>    mov    edi, 0x4006e0
   0x400623 <main+45>    mov    eax, 0
   0x400628 <main+50>    call   __isoc99_scanf@plt            <0x4004e0>
[────────────STACK────────────]
00:0000│ rsp  0x7fffffffdeb0 —▸ 0x7fffffffdfa0 ◂— 0x1
01:0008│      0x7fffffffdeb8 ◂— 0x5a6d69cf5878bc00
02:0010│ rbp  0x7fffffffdec0 —▸ 0x400650 (__libc_csu_init) ◂— push   r15
03:0018│      0x7fffffffdec8 —▸ 0x7ffff7a2d830 (__libc_start_main+240) ◂— mov    edi, eax
04:0020│      0x7fffffffded0 ◂— 0x0
05:0028│      0x7fffffffded8 —▸ 0x7fffffffdfa8 —▸ 0x7fffffffe2bb ◂— 0x65642f656d6f682f ('/home/de')
06:0030│      0x7fffffffdee0 ◂— 0x1f7ffcca0
07:0038│      0x7fffffffdee8 —▸ 0x4005f6 (main) ◂— push   rbp
[───────────BACKTRACE───────────]
 ► f 0           400612 main+28
   f 1     7ffff7a2d830 __libc_start_main+240
Breakpoint *0x400612
pwndbg> bl
Num     Type           Disp Enb Address            What
1       breakpoint     keep n   0x00000000004005f6 <main>
2       breakpoint     keep y   0x0000000000400612 <main+28>
	breakpoint already hit 1 time
3       breakpoint     keep y   0x0000000000400628 <main+50>
pwndbg> 
\end{lstlisting}

\texttt{pwndbg} is packed with other neat commands that we'll go over as we come
to the relevant exploits. It's worth spending some time getting comfortable
working in \texttt{pwndbg}, as it's a very powerful tool for developing
exploits.

Below is a list of a few useful \texttt{pwndbg} commands. These are more than
enough to get started using \texttt{pwndbg}.

% TODO: find a place for stripped binaries (info files for entry point, main
% address from _start args
% TODO: use a binary with arguments to show argc / argv / telescope
\begin{itemize}
    \item context [registers / disassemmbly / stack / backtrace] - display
        information on the program state
    \item emulate / nearpc [address] - view disassembly near address or near the
        program counter by default
    \item argc - display number of program arguments
    \item argv - display program arguments
    \item db - display data as bytes
    \item dw - display data as words
    \item dd - display data as double words
    \item dq - display data as quad words
    % NOTE: ds is buggy on both dev and stable branches
    \item ds - display data as strings
    \item dps - resolve pointers
    \item hexdump [address] - display hexdump of data at address
    \item bp - set breakpoint
    \item bd - disable breakpoint
    \item be - enable breakpoint
    \item bc - clear breakpoint
    \item bl - list breakpoints
\end{itemize}
